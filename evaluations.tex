\documentclass[11pt]{article} % Default font size is 10 pt, it can be changed here
\usepackage[top=1in, bottom=1in, left=1in, right=1in]{geometry}

\usepackage[ttscale=.875]{libertine} \usepackage[T1]{fontenc}

\PassOptionsToPackage{usenames,dvipsnames,svgnames}{xcolor}

\usepackage{adjustbox}    % Auto-resize table content (eg Opo SenSys'14 rel)
\usepackage{amsfonts}     % Adds math fonts, commands such as \begin{align}
\usepackage{amsmath}      % Provides align* environment
\usepackage{array}        % Tables for use in math mode
\usepackage{balance}      % For balanced columns on the last page
\usepackage{booktabs}     % Elegant table-formatting library
\usepackage{bold-extra}   % Provides bf+sc (only in textbf+textsc env.)
\usepackage{bytefield}    % Formatting and layout of packets / bytefields
\usepackage[skip=5pt]{caption}
%\usepackage{colortbl}     % Color table cells
\usepackage{comment}      % Provides \begin,\end{comment} for large blocks
\usepackage{cprotect}     % Allows verbatim, other formatting in macro args
\usepackage[ampersand]{easylist} % Simpler list syntax
\usepackage{endnotes}     % Footnotes pushed to the end of a document
\usepackage{enumitem}     % Allow customizations of itemize/enumerate env's
\usepackage{environ}      % Necessary for the placefigure creation
\usepackage{float}        % Allow use of [H] to force figure placement
\usepackage{gensymb}      % Adds useful symbols w/out math mode, e.g. \degree
\usepackage{graphicx}     % For importing graphics
\usepackage{hyphenat}     % Hyphenation that can break lines
\usepackage{lipsum}       % For generating temporary filler text
\usepackage{listings}     % in-line source code (poorly, consider minted)
\usepackage{makecell}     % multirow cells
\usepackage{marginnote}   % For making notes in the margin
\usepackage{mathtools}    % amsmath extension, adds more math formatting
\usepackage[protrusion=true,expansion=true,kerning,spacing]{microtype} % better type, spacing
\usepackage{multirow}     % Multiple row spacing in tables
\usepackage{nth}          % Typeset 33rd correctly as \nth{33}
%\usepackage[section]{placeins} % Don't let figs escape their sections
\usepackage{rotating}     % Rotates any object, note sideways != sidewaysfigure
%\usepackage[all=normal]{savetrees} % For when space is tight, read manual and
                          % selectively enable things. CAN BREAK CONF STYLES!!
%\usepackage{siunitx}      % SI units and significant figures
\usepackage{soul}         % Provides \hl{} for highlighting
\usepackage[subrefformat=parens]{subcaption}   % Replaces both subfig and subfigure
\usepackage{tabularx}     % Complicated table creation
\usepackage{textcomp}     % Provides \textmu for upright mu's
\usepackage{threeparttable} % Add footnotes to a table
\usepackage{tikz}         % Because drawing with text commands seems like a good idea
\usepackage{units}        % For nice fractions, \nicefrac{1}{2} --> 1/2
\usepackage{url}          % Pretty printing of hyperlinks
\usepackage[usenames,dvipsnames,svgnames]{xcolor} % Allow the use and definition of colors
\usepackage{xspace}
\usepackage{xfrac}

% Stop microtype from complaining
\microtypecontext{spacing=nonfrench}

% https://tex.stackexchange.com/questions/74170/have-new-line-between-paragraphs-no-indentation
\usepackage[parfill]{parskip}

\usepackage{setspace}
\singlespacing

% Must be last imports
\usepackage[colorlinks=true,citecolor=violet,urlcolor=Navy,linkcolor=black]{hyperref}     % Creates hyperlinks from ref/cite
\hypersetup{pdfstartview=FitH} % Sets default zoom to 100% width
\usepackage[capitalise,nameinlink,noabbrev]{cleveref}     % Do the right thing with fig/table references

% Stylize footnotes
\makeatletter
% Default:
% \def\@makefnmark{\hbox{\@textsuperscript{\normalfont\@thefnmark}}}
\renewcommand{\@makefnmark}{\makebox{\normalfont[\textcolor{red}{\@thefnmark}]}}
\renewcommand\@makefntext[1]{%
  \parindent 1em\noindent
            \hb@xt@1.8em{%
              \hss\normalfont\textcolor{red}{\@thefnmark}}~#1}
\makeatother


\graphicspath{{./img/}} % Specifies the directory where pictures are stored

% No space between bibliography items:
\let\oldthebibliography=\thebibliography
  \let\endoldthebibliography=\endthebibliography
  \renewenvironment{thebibliography}[1]{%
    \begin{oldthebibliography}{#1}%
      \setlength{\parskip}{0ex}%
      \setlength{\itemsep}{0.5ex}%
  }%
  {%
    \end{oldthebibliography}%
  }

% Some macros that a broadly useful:
\newcommand{\uW}{{\textmu}W\xspace}
\newcommand{\uw}{{\textmu}W\xspace}
\newcommand{\uA}{{\textmu}A\xspace}
\newcommand{\uV}{{\textmu}V\xspace}
\newcommand{\um}{{\textmu}m\xspace}
\newcommand{\us}{{\textmu}s\xspace}
\newcommand{\uF}{{\textmu}F\xspace}
\newcommand{\uJ}{{\textmu}J\xspace}
\newcommand{\iic}{I$^2$C\xspace}
\newcommand{\vdd}{V$_{\textnormal{DD}}$\xspace}
\newcommand{\mm}{m\textsuperscript{2}\xspace}

% Break URLs properly (thanks to Alex Halderman)
\def\UrlBreaks{\do-\do\.\do\@\do\\\do\!\do\_\do\|\do\;\do\>\do\]\do\)\do\,\do\?\do\'\do+\do\=\do\#}
\def\UrlBigBreaks{\do\:\do\/}

% Don't typset URLs in tt font
\urlstyle{sf}

\begin{document}

% No number first page
\thispagestyle{empty}

\begin{center}
  \begin{tabular*}{\textwidth}{l @{\extracolsep{\fill}} c @{\extracolsep{\fill}} r}
    \large \textbf{\textsf{ Summary of Teaching Evaluations }} &
    \large \textbf{\textsf{ Branden Ghena }} &
    \large \textbf{\textsf{ brghena@berkeley.edu }} \\
    \toprule
  \end{tabular*}
\end{center}

I've been lucky to have been involved in many different aspects of teaching.
I've taught 15-student labs, 30-student discussion sections, and 200-student
lectures. I've also designed material for homework, labs, projects, exams, and
lectures. A summary of each class, as well as selected reviews from students
are included below. The additional pages of this document are the unabridged
course evaluations I have received.

\bigskip
{\large \textit{Summer 2019}: \textbf{CS61C - Great Ideas in Computer Architecture}}

In the summer of 2019, I was one of three instructors of record on CS61C at UC
Berkeley.
%
The course had 200 students, 8 TAs, 8 tutors, and included 26 lectures and 4
projects over a compressed 8-week schedule.
%
I taught one-third of the lectures, designed exams, updated projects and
homeworks, managed course staff, and held office hours.

\hspace{\leftmargin}\textbf{Selected Quotes}
\setlist{nolistsep}
\begin{itemize}[noitemsep]
  \item My favorite lecturer by far! Clear he has a lot of knowledge but does a great job of connecting those important dots! Clear. Concise. and good depth of subject, unlike the other lecturers.
  \item I passed by and saw you talking to another student. I thought that was really caring in telling her what to do and how to go forward. Appreciate the care you have with us!
\end{itemize}


\bigskip
{\large \textit{Fall 2018}: \textbf{EE149/249A - Introduction to Embedded Systems}}

I was a TA for the embedded systems course taught by my advisor at UC Berkeley.
%
Before the semester began, I realized that a redesign of the labs would be
necessary to meet the goals of the course. I designed the Berkeley Buckler
(\url{https://github.com/lab11/buckler}) and created a new six-week lab
curriculum based on it to teach students skills that would enable them to
succeed at the two-month, open-ended design project that wraps up the course.
%
During the semester, I managed these labs, held office hours, advised project
groups, graded assignments and occasionally guest lectured.
%
I was awarded an Outstanding GSI Award from the EECS department for my efforts
in the course.

\hspace{\leftmargin}\textbf{Selected Quotes}
\setlist{nolistsep}
\begin{itemize}[noitemsep]
  \item woohoo!
  \item Very good balance when answering questions. Can go into detail, but starts with what is necessary for the course. 10/10 great TA
\end{itemize}


\bigskip
{\large \textit{Fall 2013}: \textbf{EECS370 - Introduction to Computer Organization}}

My first semester of grad school, I was a TA for EECS370 at Michigan.
%
I taught two discussion sections per week, held office hours, and graded
student assignments.
%
I was awarded an Outstanding GSI Award from the EECS department for my efforts
in the course.

\hspace{\leftmargin}\textbf{Selected Quotes}
\setlist{nolistsep}
\begin{itemize}[noitemsep]
  \item Branden may be one of the best discussion leaders I have ever had. He truly cared about the topic as well as his students. He would go out of his way to help his students understand the topics. It is clear he put in more effort than other GSI's to helping out students. I cannot stress how well Branden was able to teach the class all of the material. Very effective discussion section.
\end{itemize}


\end{document}

