\documentclass[11pt]{article} % Default font size is 10 pt, it can be changed here
\usepackage[top=1in, bottom=1in, left=1in, right=1in]{geometry}

\usepackage[ttscale=.875]{libertine} \usepackage[T1]{fontenc}

\PassOptionsToPackage{usenames,dvipsnames,svgnames}{xcolor}

\usepackage{adjustbox}    % Auto-resize table content (eg Opo SenSys'14 rel)
\usepackage{amsfonts}     % Adds math fonts, commands such as \begin{align}
\usepackage{amsmath}      % Provides align* environment
\usepackage{array}        % Tables for use in math mode
\usepackage{balance}      % For balanced columns on the last page
\usepackage{booktabs}     % Elegant table-formatting library
\usepackage{bold-extra}   % Provides bf+sc (only in textbf+textsc env.)
\usepackage{bytefield}    % Formatting and layout of packets / bytefields
\usepackage[skip=5pt]{caption}
%\usepackage{colortbl}     % Color table cells
\usepackage{comment}      % Provides \begin,\end{comment} for large blocks
\usepackage{cprotect}     % Allows verbatim, other formatting in macro args
\usepackage[ampersand]{easylist} % Simpler list syntax
\usepackage{endnotes}     % Footnotes pushed to the end of a document
\usepackage{enumitem}     % Allow customizations of itemize/enumerate env's
\usepackage{environ}      % Necessary for the placefigure creation
\usepackage{float}        % Allow use of [H] to force figure placement
\usepackage{gensymb}      % Adds useful symbols w/out math mode, e.g. \degree
\usepackage{graphicx}     % For importing graphics
\usepackage{hyphenat}     % Hyphenation that can break lines
\usepackage{lipsum}       % For generating temporary filler text
\usepackage{listings}     % in-line source code (poorly, consider minted)
\usepackage{makecell}     % multirow cells
\usepackage{marginnote}   % For making notes in the margin
\usepackage{mathtools}    % amsmath extension, adds more math formatting
\usepackage[protrusion=true,expansion=true,kerning,spacing]{microtype} % better type, spacing
\usepackage{multirow}     % Multiple row spacing in tables
\usepackage{nth}          % Typeset 33rd correctly as \nth{33}
%\usepackage[section]{placeins} % Don't let figs escape their sections
\usepackage{rotating}     % Rotates any object, note sideways != sidewaysfigure
%\usepackage[all=normal]{savetrees} % For when space is tight, read manual and
                          % selectively enable things. CAN BREAK CONF STYLES!!
%\usepackage{siunitx}      % SI units and significant figures
\usepackage{soul}         % Provides \hl{} for highlighting
\usepackage[subrefformat=parens]{subcaption}   % Replaces both subfig and subfigure
\usepackage{tabularx}     % Complicated table creation
\usepackage{textcomp}     % Provides \textmu for upright mu's
\usepackage{threeparttable} % Add footnotes to a table
\usepackage{tikz}         % Because drawing with text commands seems like a good idea
\usepackage{units}        % For nice fractions, \nicefrac{1}{2} --> 1/2
\usepackage{url}          % Pretty printing of hyperlinks
\usepackage[usenames,dvipsnames,svgnames]{xcolor} % Allow the use and definition of colors
\usepackage{xspace}
\usepackage{xfrac}

% Stop microtype from complaining
\microtypecontext{spacing=nonfrench}

% https://tex.stackexchange.com/questions/74170/have-new-line-between-paragraphs-no-indentation
\usepackage[parfill]{parskip}

\usepackage{setspace}
\singlespacing

% Must be last imports
\usepackage[colorlinks=true,citecolor=violet,urlcolor=Navy,linkcolor=black]{hyperref}     % Creates hyperlinks from ref/cite
\hypersetup{pdfstartview=FitH} % Sets default zoom to 100% width
\usepackage[capitalise,nameinlink,noabbrev]{cleveref}     % Do the right thing with fig/table references

% Stylize footnotes
\makeatletter
% Default:
% \def\@makefnmark{\hbox{\@textsuperscript{\normalfont\@thefnmark}}}
\renewcommand{\@makefnmark}{\makebox{\normalfont[\textcolor{red}{\@thefnmark}]}}
\renewcommand\@makefntext[1]{%
  \parindent 1em\noindent
            \hb@xt@1.8em{%
              \hss\normalfont\textcolor{red}{\@thefnmark}}~#1}
\makeatother


\graphicspath{{./img/}} % Specifies the directory where pictures are stored

% No space between bibliography items:
\let\oldthebibliography=\thebibliography
  \let\endoldthebibliography=\endthebibliography
  \renewenvironment{thebibliography}[1]{%
    \begin{oldthebibliography}{#1}%
      \setlength{\parskip}{0ex}%
      \setlength{\itemsep}{0.5ex}%
  }%
  {%
    \end{oldthebibliography}%
  }

% Some macros that a broadly useful:
\newcommand{\uW}{{\textmu}W\xspace}
\newcommand{\uw}{{\textmu}W\xspace}
\newcommand{\uA}{{\textmu}A\xspace}
\newcommand{\uV}{{\textmu}V\xspace}
\newcommand{\um}{{\textmu}m\xspace}
\newcommand{\us}{{\textmu}s\xspace}
\newcommand{\uF}{{\textmu}F\xspace}
\newcommand{\uJ}{{\textmu}J\xspace}
\newcommand{\iic}{I$^2$C\xspace}
\newcommand{\vdd}{V$_{\textnormal{DD}}$\xspace}
\newcommand{\mm}{m\textsuperscript{2}\xspace}

% Break URLs properly (thanks to Alex Halderman)
\def\UrlBreaks{\do-\do\.\do\@\do\\\do\!\do\_\do\|\do\;\do\>\do\]\do\)\do\,\do\?\do\'\do+\do\=\do\#}
\def\UrlBigBreaks{\do\:\do\/}

% Don't typset URLs in tt font
\urlstyle{sf}

\begin{document}

% No number first page
\thispagestyle{empty}

\begin{center}
  \begin{tabular*}{\textwidth}{l @{\extracolsep{\fill}} c @{\extracolsep{\fill}} r}
    \large \textbf{\textsf{ Diversity Statement }} &
    \large \textbf{\textsf{ Branden Ghena }} &
    \large \textbf{\textsf{ brghena@berkeley.edu }} \\
    \toprule
  \end{tabular*}
\end{center}

I hope to be seen as an ally, as someone students can go to for help and trust
to understand their problems. This isn't a simple thing to do. Based on my own
background, I come bundled with assumptions about what it's like to be a
student and what you need to do to succeed. I believe that I may never
completely understand the challenges some students face. To overcome this, I
need to listen and learn from my students and my colleagues. There are also
actions that I can take in my teaching to support students and create an
inclusive environment.

\textbf{\textsf{\large Giving Students Equal Chances to Succeed.}}\\
One of my concerns as a teacher are the implicit biases I bring with me,
particularly in grading where points need to be awarded fairly. Having clear
course policies to start with helps ensure that I am treating all students
equally. Keeping assignments anonymous while grading helps me to ensure that
I'm applying rubrics in the same manner to all students. I've found modern
tools to be very useful for this, and I use Gradescope in all of my classes to
avoid knowledge of which student's work I'm grading.

Inequity can also occur in the prior knowledge students bring with them to
class. Understanding command line tools, version control systems, and build
processes is necessary for success in many CS courses, but students are
frequently assumed to already know these skills or else are expected to learn
them as they work, disadvantaging students without prior experience in CS. I'm
particularly inspired by institutional efforts to address these issues. The
Computer Science Pragmatics[cite] course at Michigan teaches tools from package
managers to autocomplete as a one-credit course. By making tools for CS a part
of the curriculum, students in higher-level classes are able to equally focus
on the new course materials rather than ancillary skills.

\textbf{\textsf{\large Supporting Diversity.}}\\
Course staff don't just serve as teachers, but also as role models for students
as they become part of the CS community. I want students to be able to identify
with instructors, which in turn means having a diverse course staff. Recent
research has suggested that some techniques used in the hiring process can help
to create a gender-balanced staff of TAs, even when drawing from an unbalanced
student population\footnotemark. They demonstrate that focusing hiring decisions on teaching
examples and interviews resulted in a more even balance of genders than basing
decisions on GPA or course grades.

\footnotetext{Kamil, Juett, and DeOrio.
``Gender-balanced TAs from an Unbalanced Student Body.''
SIGCSE'19. 2019.}

Encouraging students, especially students from diverse backgrounds, to get
involved in extracurricular CS activities benefits everyone. As a grad student,
I mentored several students who had formerly been in classes I taught. Four of
these students are now working on PhDs in CS, including two women. Providing
opportunities to get more deeply involved in research or teaching can help
students to gain a sense of belonging in the field.

\textbf{\textsf{\large Future Efforts.}}\\
As a teacher and a voice in the department, it will be more important than ever
to be an ally. I intend to bring the lessons I have learned to my future
classes, providing diverse role models for my students and supporting equal
opportunities for learning. However, the techniques and ideas that have
resonated with me are only a part of what need to be broad efforts to support
diversity, equity, and inclusion in CS and engineering. I look forward to
listening and learning from my colleagues and my students to better understand
the challenges they are facing and to make sure that all students feel welcome
in the CS community.


\end{document}

