\documentclass[11pt]{article} % Default font size is 10 pt, it can be changed here
\usepackage[top=1in, bottom=1in, left=1in, right=1in]{geometry}

\usepackage[ttscale=.875]{libertine} \usepackage[T1]{fontenc}

\PassOptionsToPackage{usenames,dvipsnames,svgnames}{xcolor}

\usepackage{adjustbox}    % Auto-resize table content (eg Opo SenSys'14 rel)
\usepackage{amsfonts}     % Adds math fonts, commands such as \begin{align}
\usepackage{amsmath}      % Provides align* environment
\usepackage{array}        % Tables for use in math mode
\usepackage{balance}      % For balanced columns on the last page
\usepackage{booktabs}     % Elegant table-formatting library
\usepackage{bold-extra}   % Provides bf+sc (only in textbf+textsc env.)
\usepackage{bytefield}    % Formatting and layout of packets / bytefields
\usepackage[skip=5pt]{caption}
%\usepackage{colortbl}     % Color table cells
\usepackage{comment}      % Provides \begin,\end{comment} for large blocks
\usepackage{cprotect}     % Allows verbatim, other formatting in macro args
\usepackage[ampersand]{easylist} % Simpler list syntax
\usepackage{endnotes}     % Footnotes pushed to the end of a document
\usepackage{enumitem}     % Allow customizations of itemize/enumerate env's
\usepackage{environ}      % Necessary for the placefigure creation
\usepackage{float}        % Allow use of [H] to force figure placement
\usepackage{gensymb}      % Adds useful symbols w/out math mode, e.g. \degree
\usepackage{graphicx}     % For importing graphics
\usepackage{hyphenat}     % Hyphenation that can break lines
\usepackage{lipsum}       % For generating temporary filler text
\usepackage{listings}     % in-line source code (poorly, consider minted)
\usepackage{makecell}     % multirow cells
\usepackage{marginnote}   % For making notes in the margin
\usepackage{mathtools}    % amsmath extension, adds more math formatting
\usepackage[protrusion=true,expansion=true,kerning,spacing]{microtype} % better type, spacing
\usepackage{multirow}     % Multiple row spacing in tables
\usepackage{nth}          % Typeset 33rd correctly as \nth{33}
%\usepackage[section]{placeins} % Don't let figs escape their sections
\usepackage{rotating}     % Rotates any object, note sideways != sidewaysfigure
%\usepackage[all=normal]{savetrees} % For when space is tight, read manual and
                          % selectively enable things. CAN BREAK CONF STYLES!!
%\usepackage{siunitx}      % SI units and significant figures
\usepackage{soul}         % Provides \hl{} for highlighting
\usepackage[subrefformat=parens]{subcaption}   % Replaces both subfig and subfigure
\usepackage{tabularx}     % Complicated table creation
\usepackage{textcomp}     % Provides \textmu for upright mu's
\usepackage{threeparttable} % Add footnotes to a table
\usepackage{tikz}         % Because drawing with text commands seems like a good idea
\usepackage{units}        % For nice fractions, \nicefrac{1}{2} --> 1/2
\usepackage{url}          % Pretty printing of hyperlinks
\usepackage[usenames,dvipsnames,svgnames]{xcolor} % Allow the use and definition of colors
\usepackage{xspace}
\usepackage{xfrac}

% Stop microtype from complaining
\microtypecontext{spacing=nonfrench}

% https://tex.stackexchange.com/questions/74170/have-new-line-between-paragraphs-no-indentation
\usepackage[parfill]{parskip}

\usepackage{setspace}
\singlespacing

% Must be last imports
\usepackage[colorlinks=true,citecolor=violet,urlcolor=Navy,linkcolor=black]{hyperref}     % Creates hyperlinks from ref/cite
\hypersetup{pdfstartview=FitH} % Sets default zoom to 100% width
\usepackage[capitalise,nameinlink,noabbrev]{cleveref}     % Do the right thing with fig/table references

% Stylize footnotes
\makeatletter
% Default:
% \def\@makefnmark{\hbox{\@textsuperscript{\normalfont\@thefnmark}}}
\renewcommand{\@makefnmark}{\makebox{\normalfont[\textcolor{red}{\@thefnmark}]}}
\renewcommand\@makefntext[1]{%
  \parindent 1em\noindent
            \hb@xt@1.8em{%
              \hss\normalfont\textcolor{red}{\@thefnmark}}~#1}
\makeatother


\graphicspath{{./img/}} % Specifies the directory where pictures are stored

% No space between bibliography items:
\let\oldthebibliography=\thebibliography
  \let\endoldthebibliography=\endthebibliography
  \renewenvironment{thebibliography}[1]{%
    \begin{oldthebibliography}{#1}%
      \setlength{\parskip}{0ex}%
      \setlength{\itemsep}{0.5ex}%
  }%
  {%
    \end{oldthebibliography}%
  }

% Some macros that a broadly useful:
\newcommand{\uW}{{\textmu}W\xspace}
\newcommand{\uw}{{\textmu}W\xspace}
\newcommand{\uA}{{\textmu}A\xspace}
\newcommand{\uV}{{\textmu}V\xspace}
\newcommand{\um}{{\textmu}m\xspace}
\newcommand{\us}{{\textmu}s\xspace}
\newcommand{\uF}{{\textmu}F\xspace}
\newcommand{\uJ}{{\textmu}J\xspace}
\newcommand{\iic}{I$^2$C\xspace}
\newcommand{\vdd}{V$_{\textnormal{DD}}$\xspace}
\newcommand{\mm}{m\textsuperscript{2}\xspace}

% Break URLs properly (thanks to Alex Halderman)
\def\UrlBreaks{\do-\do\.\do\@\do\\\do\!\do\_\do\|\do\;\do\>\do\]\do\)\do\,\do\?\do\'\do+\do\=\do\#}
\def\UrlBigBreaks{\do\:\do\/}

% Don't typset URLs in tt font
\urlstyle{sf}

\begin{document}

% No number first page
\thispagestyle{empty}

\begin{center}
  \begin{tabular*}{\textwidth}{l @{\extracolsep{\fill}} c @{\extracolsep{\fill}} r}
    \large \textbf{\textsf{ Teaching Statement }} &
    \large \textbf{\textsf{ Branden Ghena }} &
    \large \textbf{\textsf{ brghena@berkeley.edu }} \\
    \toprule
  \end{tabular*}
\end{center}


%% CV repetition
I am fortunate to have been able to experience many facets of what it means to
be a teacher. I've been an instructor of record for a 200 student course,
taught small discussions, managed lab sections, and mentored nine individual
undergraduate researchers. As a TA for the embedded systems class, I recognized
problems with the lab curriculum and redesigned it, creating a new hardware and
software platform to improve student learning~\footnotemark.

\footnotetext{The Berkeley Buckler - \url{https://github.com/lab11/buckler}}

%% Personal desire to teach
What I discovered through my experiences is that I really love working with
students. I'm the kind of teacher that stays long after class has ended to
answer questions and explore ideas.
%
I'm excited by the concepts I teach and I want to encourage my students to
discover same excitement for themselves.
%
When students ask questions that take the ideas we've discussed and extend them
in a way I've never considered before, that's a success for me.
%I particularly look forward
%to questions from students that take the ideas we've been discussing and extend
%them in a way I've never considered before. 
%
Teaching has become a core part of who I am, and I never want to stop.
%
%helping students at office hours leaves me energized. When preparing material
%for an upcoming class, I'm excited for the new ideas students are about to
%discover.

%% Overview of philosophy
Along the way, I've also found some concepts that are important to me as an
instructor.
%
I encourage students to \textit{explore computing in a hands-on way} both
inside and out of the classroom, and my classes include projects
where students can both engage with new material and design unique solutions to
a problem.
%
To help students be willing to explore and question, I want to \textit{create
an inclusive environment} for my students where everyone is welcome.
%
In my teaching, I frequently \textit{model thought processes} that I use when
answering questions as an example for my students.
%
Through these efforts, I hope to \textit{grow a capability for self-reliance}
in my students. I don't just want them to understand the topics I teach, but
also to develop an ability to reason through possible answers to questions
they've never seen.

%% What I want to teach
%\bigskip
%\textbf{\textsf{\large Teaching Interests.}}\\
\textbf{Teaching Interests.}
My experiences as an embedded systems researcher have given me a broad set of
skills across the hardware and software systems domains that I am excited to
share with my students. I am particularly eager to teach:
%
\begin{itemize}
  \item Introductory systems such as computer organization or systems programming
  \item Introductory programming or data structures
  \item Upper-level systems such as computer architecture or operating systems
  \item Embedded systems at the introductory or advanced level
\end{itemize}
%
I have experience in designing embedded systems curriculum and am looking
forward to creating courses that connect computation and the real-world
together. These could be focused towards the introductory or upper levels
depending on the needs of the department.

%I am particularly eager to teach introductory systems classes such as computer
%organization or systems programming that connect multiple layers of the stack.
%%
%I have experience in designing embedded systems curriculum, and am looking
%forward to creating and teaching a course that connects real-world and
%computing together, either at the introductory or upper level depending on the
%needs of the department.
%%
%Teaching introductory programming or data structures courses also
%greatly interest me as an opportunity to share my passion for computing with
%students exploring what it means to be a computer scientist.
%%
%At the upper-division level, classes in \textit{computer architecture and
%operating systems} are good fits for my prior experience and I would be 

%\textbf{\textsf{\large Teaching Interests.}}\\
%My experiences in embedded systems have resulted in a wide set of skills and
%knowledge that span software and hardware systems, and I am comfortable
%teaching a variety of computer science and engineering courses. I am interested
%in teaching any courses in the domain of computer science and engineering,
%especially:
%%
%\setlist{nolistsep}
%\begin{itemize}[noitemsep]
%  \item Introductory CS courses: Introduction to Programming and Data Structures
%  \item Introductory EE/CE courses: Introduction to Engineering and Digital Logic
%  \item Mid-Level Systems courses: Computer Organization and Systems Programming
%  \item Upper-Level Systems courses: Embedded Systems, Operating Systems, and Computer Architecture
%\end{itemize}
%
%I am particularly interested in teaching or developing a project-based embedded
%systems course that encourages students to explore connections between
%computing and the real-world.


\bigskip
\textbf{\textsf{\large Promoting Hands-On Learning.}}\\
As both a student and an instructor, my experience has been that hands-on
learning is critical to forming a deep understanding of a concept. In the
classroom, I use projects to allow students to engage with new material while
also exercising secondary skills like debugging. I've found that good projects
give students the freedom to design and implement their own solution to a
problem. Whenever possible, I prefer to leave the problem up to the students as
well. My role as the instructor is then to advise how the lessons of the class
can be combined with students' desired goals and to use my experience to guide
them away from possible pitfalls. The embedded systems class at UC Berkeley
ends with a two-month, open-ended project where students, with the guidance of
the TAs, can work on whatever topic they desire as long as it can be related
back to course concepts. Allowing choice and ownership in the project increases
student motivation~\footnotemark{} and enables them to connect to passions outside
of the EECS world. In past semesters students have integrated ideas
from industrial engineering, civil engineering, and music theory in their
semester projects.

\footnotetext{Blumenfeld, Soloway, Marx, Krajcik, Guzdial, and Palincsar.
``Motivating Project-Based Learning: Sustaining the Doing, Supporting the Learning.''
Educational Psychologist. 1991.}

Advocating for hands-on learning extends outside of my own classroom as well.
%Learning occurs outside of the classroom as well,
%and my background in embedded
%systems leaves me well-situated to encourage it.
The ability to connect computation and the real world together is valuable to
students in a wide variety of fields---far more than will realistically enroll
in any embedded systems course.
%
There ends up being a gap between goals students have for a project and their
knowledge on how to achieve them.
I see bridging this gap as part of my role as an instructor.
%
At Michigan, I held cyber-physical systems office hours that were not connected
to any class, but were instead open to the entire university. They were
frequently attended by student groups and senior-design teams throughout the
engineering disciplines, and even occasionally by other researchers.
%
Based on their needs and goals,
I pointed groups to existing hardware and software tools that could help in
their task, taught them skills like soldering or circuit board design, and
generally provided advice about possible problems they would encounter.
%
By providing a public resource that could fill in missing expertise for student
groups, open office hours encourage more hands-on projects that combine
computing and the real-world. I intend to continue the practice of open office
hours and explore how it could be extended to other domains of computer science
and engineering.


\bigskip
\textbf{\textsf{\large Creating an Inclusive Environment.}}\\
Everyone is welcome in my classroom. I want all students to feel that they can
be a part of the STEM community.
%
To support this, I find it important to have a diverse course staff in terms
of gender and racial identity as well as education background. My goal is
that all students should have multiple role models they can identify with.
%
This can be difficult to accomplish, depending on the size of your staff and the
disparity in the student population, but recent research has suggested that
some techniques used in the hiring process can help to create a gender-balanced
staff of TAs, even when drawing from an unbalanced student
population~\footnotemark{}. They demonstrate that focusing hiring decisions on
teaching examples and interviews resulted in a more even balance of genders
than basing decisions on GPA or course grades, a practice I plan to follow.

\footnotetext{Kamil, Juett, and DeOrio.
``Gender-balanced TAs from an Unbalanced Student Body.''
SIGCSE'19. 2019.}

Another aspect of creating an inclusive environment is to allow student voices
to be equally heard, and I try to provide many opportunities for students to
contribute.
%
Discussion in class is great, but many students are unwilling to speak up or
possibly be wrong in front of their peers, especially if they feel like they
may not belong.
%
I encourage students to feel free to talk with me after class, post on a class
forum, or bring questions to office hours if they want a private discussion. I
also try to watch out for students whose voices are being unintentionally
overtaken by other more confident students so that I can pause the discussion
and give them a chance to be heard.
%
Despite this, I've felt sometimes in my own teaching that a minority of student
voices are heard. Practices from Harvey Mudd suggest that framing the
classroom climate by explicitly addressing the pedagogical value of wrong
answers can help encourage involvement, while techniques like random selection
of students (via a deck of cards in the example) enforces interactions from
everyone without unfairly targetting anyone~\footnotemark{}.
%
%All of engineering benefits from diverse voices and I want to encourage them
%whenever possible.

\footnotetext{Barker, O'Neill, and Kazim.
``Framing classroom climate for student learning and retention in computer science.''
SIGCSE'14. 2014.}

\bigskip
\textbf{\textsf{\large Modeling Thought Processes.}}\\
Becoming an engineer isn't just about learning facts, but also about a
different process of thinking that I want to teach to my students. Questions in
class that I don't know the answer to are a wonderful opportunity to
demonstrate that engineering thought process. First, I acknowledge that I'm
unsure what the true answer is. I want students to understand that everyone has
gaps in their knowledge and that facing uncertainty is a normal occurrence.
Next, I verbally walk through my thought process on what the answer could be
and why. Details like simplifying assumptions I'm making and additional
knowledge I'm applying to the problem serve as examples to students for how
they can reason through future engineering questions they will face.

When teaching programming, live coding is useful in demonstrating the process
to students~\footnotemark{}. I've been programming for fifteen years, but I still
consult the internet to remember API details, and I often write python code
with an interpreter running in another terminal so I can test out small code
snippets as I go. None of this is surprising to someone who has experience
coding, but these parts of the process rarely make it onto lecture slides. Live
demonstrations give students insight into the types of little questions that
programmers ask themselves as a part of their process and the techniques we use
to answer them.

\footnotetext{Marc J.\ Rubin.
``The Effectiveness of Live-Coding to Teach Introductory Programming.''
SIGCSE'13. 2013.}

%When I teach a class, I want to explain not just the subject to my students,
%but also why it is important. I find this to be especially important for
%cross-disciplinary topics.
I also explain the thinking that went into the design of my course materials
to my students.
%
Most of my teaching experiences have been in courses that connect hardware and
software together, and I've found that software-minded students are often
unsure of why they should bother learning hardware details. Addressing that
question directly in class can help to place new material in the proper context
and motivate student interest in the subject.
%
One of the topics I taught as an instructor for the computer organization class
at UC Berkeley was digital logic. I began the lecture with an introduction on
why this material was important, noting the lessons hardware teaches us about
higher-layer software's reliability, performance, and security.


\bigskip
\textbf{\textsf{\large Encouraging Self-Reliance.}}\\
As a teacher, I want to encourage my students to have the skills and confidence
to independently solve the problems they encounter.
%
This was a particular goal when I redesigned the labs for the embedded systems
course at UC Berkeley.
%
While the existing lab curriculum successfully taught many important concepts,
students often relied heavily on TA support and weren't building confidence in
their own abilities.

I realized that a key issue with the embedded systems labs was that students
did not feel in control of their own success due to overly complicated and
unreliable tools, particularly the development environment used to build,
upload, and debug embedded software. To support a diverse range of hardware,
the environment included many configuration options, the majority of which were
not relevant to our labs and would cause errors. Tracking these down to a
specific source was difficult, and students ended up relying heavily on TA
support as they didn't understand the steps to the build process that the
environment was running. The problems weren't bad enough to prevent students
from completing lab assignments, but they led to a feeling of helplessness
that defeated the lab's purpose.

The solution was to return a sense of control to the students by
reducing the tools into simple, understandable steps. While past generations of
TAs had focused on removing errors and documenting complexity, I instead
replaced the single development environment with smaller tools that were
manually invoked by the students to handle building, uploading, and debugging
code. The total number of errors students faced during the course of a lab was
probably not reduced by the changes, but the impact was very different. Now
that nothing attempted to handle the entire process opaquely, errors were
easier to attribute to a single step and then resolve, and because the students
were manually running each tool, they felt that they had agency over the
outcome and would attempt to fix problems themselves.


\end{document}

