\documentclass[11pt]{article} % Default font size is 10 pt, it can be changed here
\usepackage[top=1in, bottom=1in, left=1in, right=1in]{geometry}

\usepackage[ttscale=.875]{libertine} \usepackage[T1]{fontenc}

\PassOptionsToPackage{usenames,dvipsnames,svgnames}{xcolor}

\usepackage{adjustbox}    % Auto-resize table content (eg Opo SenSys'14 rel)
\usepackage{amsfonts}     % Adds math fonts, commands such as \begin{align}
\usepackage{amsmath}      % Provides align* environment
\usepackage{array}        % Tables for use in math mode
\usepackage{balance}      % For balanced columns on the last page
\usepackage{booktabs}     % Elegant table-formatting library
\usepackage{bold-extra}   % Provides bf+sc (only in textbf+textsc env.)
\usepackage{bytefield}    % Formatting and layout of packets / bytefields
\usepackage[skip=5pt]{caption}
%\usepackage{colortbl}     % Color table cells
\usepackage{comment}      % Provides \begin,\end{comment} for large blocks
\usepackage{cprotect}     % Allows verbatim, other formatting in macro args
\usepackage[ampersand]{easylist} % Simpler list syntax
\usepackage{endnotes}     % Footnotes pushed to the end of a document
\usepackage{enumitem}     % Allow customizations of itemize/enumerate env's
\usepackage{environ}      % Necessary for the placefigure creation
\usepackage{float}        % Allow use of [H] to force figure placement
\usepackage{gensymb}      % Adds useful symbols w/out math mode, e.g. \degree
\usepackage{graphicx}     % For importing graphics
\usepackage{hyphenat}     % Hyphenation that can break lines
\usepackage{lipsum}       % For generating temporary filler text
\usepackage{listings}     % in-line source code (poorly, consider minted)
\usepackage{makecell}     % multirow cells
\usepackage{marginnote}   % For making notes in the margin
\usepackage{mathtools}    % amsmath extension, adds more math formatting
\usepackage[protrusion=true,expansion=true,kerning,spacing]{microtype} % better type, spacing
\usepackage{multirow}     % Multiple row spacing in tables
\usepackage{nth}          % Typeset 33rd correctly as \nth{33}
%\usepackage[section]{placeins} % Don't let figs escape their sections
\usepackage{rotating}     % Rotates any object, note sideways != sidewaysfigure
%\usepackage[all=normal]{savetrees} % For when space is tight, read manual and
                          % selectively enable things. CAN BREAK CONF STYLES!!
%\usepackage{siunitx}      % SI units and significant figures
\usepackage{soul}         % Provides \hl{} for highlighting
\usepackage[subrefformat=parens]{subcaption}   % Replaces both subfig and subfigure
\usepackage{tabularx}     % Complicated table creation
\usepackage{textcomp}     % Provides \textmu for upright mu's
\usepackage{threeparttable} % Add footnotes to a table
\usepackage{tikz}         % Because drawing with text commands seems like a good idea
\usepackage{units}        % For nice fractions, \nicefrac{1}{2} --> 1/2
\usepackage{url}          % Pretty printing of hyperlinks
\usepackage[usenames,dvipsnames,svgnames]{xcolor} % Allow the use and definition of colors
\usepackage{xspace}
\usepackage{xfrac}

% Stop microtype from complaining
\microtypecontext{spacing=nonfrench}

% https://tex.stackexchange.com/questions/74170/have-new-line-between-paragraphs-no-indentation
\usepackage[parfill]{parskip}

\usepackage{setspace}
\singlespacing

% Must be last imports
\usepackage[colorlinks=true,citecolor=violet,urlcolor=Navy,linkcolor=black]{hyperref}     % Creates hyperlinks from ref/cite
\hypersetup{pdfstartview=FitH} % Sets default zoom to 100% width
\usepackage[capitalise,nameinlink,noabbrev]{cleveref}     % Do the right thing with fig/table references

% Stylize footnotes
\makeatletter
% Default:
% \def\@makefnmark{\hbox{\@textsuperscript{\normalfont\@thefnmark}}}
\renewcommand{\@makefnmark}{\makebox{\normalfont[\textcolor{red}{\@thefnmark}]}}
\renewcommand\@makefntext[1]{%
  \parindent 1em\noindent
            \hb@xt@1.8em{%
              \hss\normalfont\textcolor{red}{\@thefnmark}}~#1}
\makeatother


\graphicspath{{./img/}} % Specifies the directory where pictures are stored

% No space between bibliography items:
\let\oldthebibliography=\thebibliography
  \let\endoldthebibliography=\endthebibliography
  \renewenvironment{thebibliography}[1]{%
    \begin{oldthebibliography}{#1}%
      \setlength{\parskip}{0ex}%
      \setlength{\itemsep}{0.5ex}%
  }%
  {%
    \end{oldthebibliography}%
  }

% Some macros that a broadly useful:
\newcommand{\uW}{{\textmu}W\xspace}
\newcommand{\uw}{{\textmu}W\xspace}
\newcommand{\uA}{{\textmu}A\xspace}
\newcommand{\uV}{{\textmu}V\xspace}
\newcommand{\um}{{\textmu}m\xspace}
\newcommand{\us}{{\textmu}s\xspace}
\newcommand{\uF}{{\textmu}F\xspace}
\newcommand{\uJ}{{\textmu}J\xspace}
\newcommand{\iic}{I$^2$C\xspace}
\newcommand{\vdd}{V$_{\textnormal{DD}}$\xspace}
\newcommand{\mm}{m\textsuperscript{2}\xspace}

% Break URLs properly (thanks to Alex Halderman)
\def\UrlBreaks{\do-\do\.\do\@\do\\\do\!\do\_\do\|\do\;\do\>\do\]\do\)\do\,\do\?\do\'\do+\do\=\do\#}
\def\UrlBigBreaks{\do\:\do\/}

% Don't typset URLs in tt font
\urlstyle{sf}

\begin{document}

% No number first page
\thispagestyle{empty}

\begin{center}
  \begin{tabular*}{\textwidth}{l @{\extracolsep{\fill}} c @{\extracolsep{\fill}} r}
    \large \textbf{\textsf{ Diversity Statement }} &
    \large \textbf{\textsf{ Branden Ghena }} &
    \large \textbf{\textsf{ brghena@berkeley.edu }} \\
    \toprule
  \end{tabular*}
\end{center}

I hope to be seen as an ally, as someone students can go to for help and as
someone they can trust to understand their problems. This isn't a simple thing
to do.
%
%Based on my own background,
I come bundled with assumptions about what it's like to be a student, but my
experiences aren't theirs. I will never completely understand what challenges
some students face.
%
To counter this, I need to listen and learn from my students and my colleagues
to better understand the challenges they are facing and to make sure that all
students feel welcome.
%
%, and I need to use the knowledge I've gathered to continuously
%improve my teaching and create an inclusive environment.
%There are also actions that I can take in my teaching to support
%students and create an inclusive environment.

\bigskip
\textbf{\textsf{\large Giving Students Equal Opportunity to Succeed.}}\\
One of my concerns as a teacher are the unconscious biases I bring with me. One
particular focus of mine is how bias can affect grading and whether points are
being awarded fairly.
%
%Having clear course policies to start with helps ensure that I am treating all
%students equally.
%
Keeping assignments anonymous while grading helps avoid allowing scoring to be
based on any information except the solution given~\footnotemark{}. I've found
modern tools to be very useful for this, and I use Gradescope in all of my
classes to avoid knowledge of which student's work I'm grading.
%
Grading cannot always be anonymous, however, especially when judging
individualized student projects.
%
As a TA for the embedded systems class at UC Berkeley, I created rigorous
rubrics for project grading to ensure that all students would be assessed by
the same criteria.

\footnotetext{David J.\ Brennan.
``University student anonymity in the summative assessment of written work.''
Higher Education Research \& Development. 2008.}

Inequity also occurs in the amount of prior knowledge students bring with them
to class.
%
Understanding command line tools, version control systems, and build processes
is necessary for success in many CS courses, but students are frequently
assumed either to already know these skills or else are expected to learn them
as they work, disadvantaging students without prior experience.
%
I'm particularly inspired by institutional efforts to address these issues.
Computer Science Pragmatics at Michigan teaches tools from package managers to
autocomplete as a one-credit required course~\footnotemark{}.
%
By making tools for CS a official part of the curriculum, students in later
classes are able to equally focus on the new course materials rather than
building ancillary skills.

\footnotetext{Computer Science Pragmatics - \url{https://cspragmatics.com/}}

\bigskip
\textbf{\textsf{\large Supporting Diversity.}}\\
Course staff don't just serve as teachers, but also as role models for their
students. They play an important role as an example of what the STEM community
is like.
%
It is important to me that my students have instructors that they can identify
with, which in turns means having a diverse course staff whenever possible.
%
Recent research has suggested that some techniques used in the hiring process
can help to create a gender-balanced staff of TAs, even when drawing from an
unbalanced student population~\footnotemark{}. In their classes, a focus on
teaching examples and interviews resulted in a more even balance of genders
among the course staff than focusing on GPA or course grades would have.

\footnotetext{Kamil, Juett, and DeOrio.
``Gender-balanced TAs from an Unbalanced Student Body.''
SIGCSE'19. 2019.}

I have also tried to support diversity in STEM through my research.
%
As a grad student, it was a priority of mine to provide opportunities for
undergraduates to get involved in research. I advised nine undergraduate
researchers at Michigan, including two women who have both continued on to
Ph.D.\ programs in computer science.
%
As I became more involved with teaching at UC Berkeley, I saw my role as
someone who could encourage students to get involved in research. I reached out
to students to encourage them and offer to answer questions about what it is
like to be a grad student researcher. Seven of my former students, including
three women, are now researching with my lab.
%
I hope that providing opportunities and encouragement to students can help them
to gain  a sense of belonging in  the field.
%Encouraging students, especially students from diverse backgrounds, to get
%involved in extracurricular CS activities benefits everyone.
%Four of
%these students are now working on Ph.D.s in CS, including two women. Providing
%opportunities to get more deeply involved in research or teaching can help
%students to gain a sense of belonging in the field.

%\bigskip
%\textbf{\textsf{\large Future Efforts.}}\\
%As a teacher and a voice in the department, it will be more important than ever
%to be an ally. I intend to bring the lessons I have learned to my future
%classes, providing diverse role models for my students and supporting equal
%opportunities for learning. However, the techniques and ideas that have
%resonated with me are only a part of what need to be broad efforts to support
%diversity, equity, and inclusion in STEM. I look forward to
%listening and learning from my colleagues and my students
%
%to better understand
%the challenges they are facing and to make sure that all students feel welcome
%in the CS community.


\end{document}

