\documentclass[11pt]{article} % Default font size is 10 pt, it can be changed here
\usepackage[top=1in, bottom=1in, left=1in, right=1in]{geometry}

\usepackage[%
  backend=biber,
  %style=numeric,
  sorting=nyt, % https://tex.stackexchange.com/questions/51434/biblatex-citation-order
  minnames=15,
  maxnames=25,
  defernumbers=true,
]{biblatex}
\setlength\bibitemsep{0.60\itemsep}

\usepackage[ttscale=.875]{libertine} \usepackage[T1]{fontenc}

\PassOptionsToPackage{usenames,dvipsnames,svgnames}{xcolor}

\usepackage{adjustbox}    % Auto-resize table content (eg Opo SenSys'14 rel)
\usepackage{amsfonts}     % Adds math fonts, commands such as \begin{align}
\usepackage{amsmath}      % Provides align* environment
\usepackage{array}        % Tables for use in math mode
\usepackage{balance}      % For balanced columns on the last page
\usepackage{booktabs}     % Elegant table-formatting library
\usepackage{bold-extra}   % Provides bf+sc (only in textbf+textsc env.)
\usepackage{bytefield}    % Formatting and layout of packets / bytefields
\usepackage[skip=5pt]{caption}
%\usepackage{colortbl}     % Color table cells
\usepackage{comment}      % Provides \begin,\end{comment} for large blocks
\usepackage{cprotect}     % Allows verbatim, other formatting in macro args
\usepackage[ampersand]{easylist} % Simpler list syntax
\usepackage{endnotes}     % Footnotes pushed to the end of a document
\usepackage{enumitem}     % Allow customizations of itemize/enumerate env's
\usepackage{environ}      % Necessary for the placefigure creation
\usepackage{float}        % Allow use of [H] to force figure placement
\usepackage{gensymb}      % Adds useful symbols w/out math mode, e.g. \degree
\usepackage{graphicx}     % For importing graphics
\usepackage{hyphenat}     % Hyphenation that can break lines
\usepackage{lipsum}       % For generating temporary filler text
\usepackage{listings}     % in-line source code (poorly, consider minted)
\usepackage{makecell}     % multirow cells
\usepackage{marginnote}   % For making notes in the margin
\usepackage{mathtools}    % amsmath extension, adds more math formatting
\usepackage[protrusion=true,expansion=true,kerning,spacing]{microtype} % better type, spacing
\usepackage{multirow}     % Multiple row spacing in tables
\usepackage{nth}          % Typeset 33rd correctly as \nth{33}
%\usepackage[section]{placeins} % Don't let figs escape their sections
\usepackage{rotating}     % Rotates any object, note sideways != sidewaysfigure
%\usepackage[all=normal]{savetrees} % For when space is tight, read manual and
                          % selectively enable things. CAN BREAK CONF STYLES!!
%\usepackage{siunitx}      % SI units and significant figures
\usepackage{soul}         % Provides \hl{} for highlighting
\usepackage[subrefformat=parens]{subcaption}   % Replaces both subfig and subfigure
\usepackage{tabularx}     % Complicated table creation
\usepackage{textcomp}     % Provides \textmu for upright mu's
\usepackage{threeparttable} % Add footnotes to a table
\usepackage{tikz}         % Because drawing with text commands seems like a good idea
\usepackage{units}        % For nice fractions, \nicefrac{1}{2} --> 1/2
\usepackage{url}          % Pretty printing of hyperlinks
\usepackage[usenames,dvipsnames,svgnames]{xcolor} % Allow the use and definition of colors
\usepackage{xspace}
\usepackage{xfrac}

% Stop microtype from complaining
\microtypecontext{spacing=nonfrench}

% https://tex.stackexchange.com/questions/74170/have-new-line-between-paragraphs-no-indentation
\usepackage[parfill]{parskip}

\usepackage{setspace}
\singlespacing

% Must be last imports
\usepackage[colorlinks=true,citecolor=red,urlcolor=Navy,linkcolor=black]{hyperref}     % Creates hyperlinks from ref/cite
\hypersetup{pdfstartview=FitH} % Sets default zoom to 100% width
\usepackage[capitalise,nameinlink,noabbrev]{cleveref}     % Do the right thing with fig/table references

% Stylize footnotes
\makeatletter
% Default:
% \def\@makefnmark{\hbox{\@textsuperscript{\normalfont\@thefnmark}}}
\renewcommand{\@makefnmark}{\makebox{\normalfont[\textcolor{red}{\@thefnmark}]}}
\renewcommand\@makefntext[1]{%
  \parindent 1em\noindent
            \hb@xt@1.8em{%
              \hss\normalfont\textcolor{red}{\@thefnmark}}~#1}
\makeatother


\graphicspath{{./img/}} % Specifies the directory where pictures are stored

% No space between bibliography items:
\let\oldthebibliography=\thebibliography
  \let\endoldthebibliography=\endthebibliography
  \renewenvironment{thebibliography}[1]{%
    \begin{oldthebibliography}{#1}%
      \setlength{\parskip}{0ex}%
      \setlength{\itemsep}{0.5ex}%
  }%
  {%
    \end{oldthebibliography}%
  }

% Some macros that a broadly useful:
\newcommand{\uW}{{\textmu}W\xspace}
\newcommand{\uw}{{\textmu}W\xspace}
\newcommand{\uA}{{\textmu}A\xspace}
\newcommand{\uV}{{\textmu}V\xspace}
\newcommand{\um}{{\textmu}m\xspace}
\newcommand{\us}{{\textmu}s\xspace}
\newcommand{\uF}{{\textmu}F\xspace}
\newcommand{\uJ}{{\textmu}J\xspace}
\newcommand{\iic}{I$^2$C\xspace}
\newcommand{\vdd}{V$_{\textnormal{DD}}$\xspace}
\newcommand{\mm}{m\textsuperscript{2}\xspace}

% Load bibs the biber way
\addbibresource{references.bib}

% Custom bibtex keys, from
% http://tex.stackexchange.com/questions/111846/biblatex-2-custom-fields-only-one-is-working
\DeclareSourcemap{
  \maps[datatype=bibtex,overwrite=true]{
    \map{
      \step[fieldsource=acceptance-total]
      \step[fieldset=usera,origfieldval]
    }
    \map{
      \step[fieldsource=acceptance-accepted]
      \step[fieldset=userb,origfieldval]
    }
    \map{
      \step[fieldsource=acceptance-total]
      \step[fieldset=acceptance-accepted,append=true,fieldvalue=/]
      \step[fieldset=acceptance-accepted,append=true,origfieldval]
      \step[fieldsource=acceptance-accepted]
      \step[fieldset=userc,append=true,origfieldval]
    }
    \map{
      \step[fieldsource=extra]
      \step[fieldset=userd,origfieldval]
    }
    \map{
      \step[fieldsource=acceptance-percent]
      \step[fieldset=usere,origfieldval]
    }
  }
}

% Acceptance rates in bib. Print Extra Info.
\ExplSyntaxOn
\NewDocumentCommand{\myMathFunction}{m}
{ \fp_eval:n {round((#1)*100)} }
\ExplSyntaxOff

\DeclareFieldFormat{userc}{\myMathFunction{#1}\%}

\AtEveryBibitem{%
  \csappto{blx@bbx@\thefield{entrytype}}{% put at end of entry
    \iffieldundef{usera}{%
%    \space \textbf{No annotation!}}{%
    }{%
      \\Acceptance:%
      \space\printfield{userb}~/~\printfield{usera}%
      \space(\printfield{userc}).
    }
    \iffieldundef{usere}{%
    }{%
      \\Acceptance:%
      \space\printfield{usere}\%.
    }
    \iffieldundef{userd}{%
      %
    }{%
      \textbf{\\\bf\printfield{userd}.}
    }
  }
}

% https://tex.stackexchange.com/questions/297087/putting-the-title-first-in-the-bibliography
\newcommand{\nameuse}[1]{%
  \def\do##1{\settoggle{blx@use##1}{#1}}%
  \dolistcsloop{blx@datamodel@names}}

\newcommand{\nameusesave}{%
  \def\do##1{%
    \providetoggle{blx@save@use##1}%
    \iftoggle{blx@use##1}{\toggletrue{blx@save@use##1}}{\togglefalse{blx@save@use##1}}%
  }%
  \dolistcsloop{blx@datamodel@names}}

\newcommand{\nameuserestore}{%
  \def\do##1{%
    \iftoggle{blx@save@use##1}{\toggletrue{blx@use##1}}{\togglefalse{blx@use##1}}%
  }%
  \dolistcsloop{blx@datamodel@names}}


% Break URLs properly (thanks to Alex Halderman)
\def\UrlBreaks{\do-\do\.\do\@\do\\\do\!\do\_\do\|\do\;\do\>\do\]\do\)\do\,\do\?\do\'\do+\do\=\do\#}
\def\UrlBigBreaks{\do\:\do\/}

% Don't typset URLs in tt font
\urlstyle{sf}

\begin{document}

% No number first page
\thispagestyle{empty}

\begin{center}
  \begin{tabular*}{\textwidth}{l @{\extracolsep{\fill}} c @{\extracolsep{\fill}} r}
    \large \textbf{\textsf{ Research Statement }} &
    \large \textbf{\textsf{ Branden Ghena }} &
    \large \textbf{\textsf{ brghena@berkeley.edu }} \\
    \toprule
  \end{tabular*}
\end{center}

I have always been drawn to the intersection of computers and the real world.
One aspect that has always drawn me to the area was the many domains of
expertise that are combined in order to create an embedded system. In the
course of my research, I've worked on circuits and websites and everything in
between.

My research focuses on two areas: investigations of low-power networking
standards and platforms for enabling future research.

\textbf{\textsf{\large Low-Power Networking Standards.}}\\ Early research in
wireless sensor networks focused on the creation of new network standards,
especially access control mechanisms. Today, many of these techniques have been
incorporated into wireless networking standards and the sensor network world
can, for the first time, consider the use of existing standards rather than the
development of new ones to meet application needs. My interest has been the
study of these existing network schemes in order to determine which
applications they can best serve and where they are still lacking.

\textbf{Low-Power Wide-Area Networks} (LPWANs) address a major deficit in prior wireless
networks: wide-area machine-to-machine communications. While solutions exist in
the local-area realm, human-centric cellular networks that focus on download
throughput have long been the only players if long-range communications are
desired. Over the last several years, however, protocols like LoRaWAN that
operate on the unlicensed 915 MHz band in the United States have grown to fill
this void. Their use of simple protocols and ability to transmit at ranges over
a kilometer while drawing only a few hundred milliwatts enables exciting new
applications. [FIGURE Range vs Throughput]

However, as we investigated these new networks, we realized that they face
problems when attempting to serve real-world applications~\cite{ghena19lpwans}. As we
began our investigations, we realized a new metric was needed. Neither range
nor throughput were sufficient metrics for studying LPWANs. We defined bit
flux, a measure of throughput over coverage area as a way of comparing them.
This metric can also be used to measure application needs which allows
suitability to be easily measured. Comparing the bit flux requirements of
several example applications and the bit flux that LPWANs provide, we find that
unlicensed LPWANs are only suitable for low-rate, sparse sensing applications. 

LPWANs are suffering from two major issues. The first is a capacity issue:
low-throughput over a very wide area results in a very limited amount of
throughput available to each deployed device. In addition to the capacity
issues, LPWANs deployed in the unlicensed bands face a coexistence problem. The
long range of these networks means that many stakeholders will overlap in
network coverage areas, especially in urban environments. We argue that these
problems must be solved in the near-term in order to allow unlicensed-band,
long-range networking to be successful.

\textbf{Bluetooth Low Energy} (BLE) is a common communication protocol in smartphones,
laptops, and other consumer products, including many Internet of Things
devices. BLE communication reduces or eliminates listening costs for energy
constrained devices, avoids interference via channel diversity, and allows
intercommunication between sensors and people through personal devices. My
explorations of BLE capabilities has focused on BLE advertisements—simple,
periodic, broadcast messages intended for device discovery. Using
advertisements, a single-hop, star-topology network can be created in full
compliance with the BLE specification that allows any number of devices to send
data to any number of gateways[cite??].

My research began by composing models that describe reception rates for BLE
advertisements. Advertisements are sent without any coordination or
channel-sense mechanism, essentially an ALOHA[cite] access control mechanism,
which leads to packet collisions as a primary mechanism of data loss.
Developing analytical models for these packet collisions allows an upper bound
of reception rate for a deployed network to be determined in advance based on
the number of deployed devices and their transmission rate. These models allow
us to recognize issues with network deployments that are due to external
factors rather than packet collisions alone. I studied a deployment of BLE
power meters[cite] that significantly underperformed expectations, finding that
problems with the gateway BLE hardware were at fault.

\textbf{Future directions.} The biggest potential for disruption in today's low-power
wireless space is the emergence of new cellular protocols targeted at
machine-to-machine communication. LTE-M and NB-IoT arose from lengthy
development by 3GPP and are finally supported by network operators in the US
and world-wide. Particularly these networks could come to dominate the nascent
LPWAN space, but they have applications to the indoor space as well. The ideal
of infrastructure-free deployments is highly motivating. From a
resource-constrained systems view, how to successfully use these protocols to
serve application needs is an open question. While they can be low energy, they
remain high power, necessitating batching of messages to maintain energy
budgets. Delay-tolerant network techniques may be helpful in order to maintain
existing network primitives while dealing with necessary high latency.

\textbf{\textsf{\large Enabling Research through Platform Development.}}\\
The embedded systems field creates many artifacts. In my own career I've
created sensors, simulations, protocols, and libraries that are capable of
accomplishing many tasks. I've always been particularly motivated by the
creation of tools that enable exploration in new domains that were previously
unapproachable.

\textbf{Signpost} aims to reduce the development burden for city-scale sensing
projects[cite] [FIGURE of architecture]. City-scale sensing holds a lot of
promise and interest, and for good reason: applications such as pedestrian
route planning based on air quality, noise pollution monitoring, and automatic
emergency response alerts can all improve the quality of life for a city's
inhabitants. Developing a new sensing system requires much more than just an
impactful idea, however. Hardware must be designed to support a new sensor with
energy, communications, storage, and processing capability. These requirements
make it particularly challenging to perform short-term, exploratory research
necessary to test out ideas in the first place.

The Signpost hardware and software platform addresses these challenges by
providing commonly required resources through a modular interface. Hosted
modules provide sensors and application processing while Signpost provides
energy, network connection, storage, time, location, and
Linux-as-a-coprocessor. In order to enable multiple simultaneous deployments,
Signpost meters shared resources such as energy and communication bandwidth in
order to provide fairness. The API for accessing these resources is abstracted
over a shared I2C bus and has been implemented for several embedded software
platforms including Tock, mbed, and Arduino.

\textbf{Tock} addresses the need for modern operating systems support on
resource-constrained, microcontroller-based systems[cite] [Figure of Tock
kernel and apps]. While microcontrollers are the base compute platform for
embedded systems and the Internet of Things, OS support for them has critically
lagged behind. Most modern embedded OSes compile a kernel and single
application into a single binary that is loaded onto the microcontroller
without any layered protection mechanisms. Support for multiprogramming is
nearly unheard of. If you only have 64 kB of RAM for the entire system sharing
it was not even a consideration.

Tock demonstrates that reliable, safe, and dynamic multiprogramming is indeed
possible on microcontrollers. Tock enforces a system-call boundary between
applications and the kernel using new hardware features that allow for
segment-based memory protection. However, even the kernel itself is composed of
many device drivers that may not all be trustworthy. We demonstrate the use of
Rust, a type-safe systems language with memory efficiency and performance close
to C, to enforce driver isolation[cite].

\textbf{Future Directions.} Platforms research benefits not only other communities who
can use the research platform as a tool, but also my own community of embedded
systems researchers by elucidating which areas remain under-developed. While
resources like memory, processor time, and even network access are commonly
shared, energy is much less widely considered. Isolation between applications
on an energy-harvesting platform is possible but requires new policies and
APIs[cite].

As I have developed platforms, I realized a trend has been occurring towards
multi-microcontroller designs. Rather than handle complex software interactions
between tasks such as networking and sensing, embedded platforms have become
distributed systems, with multiple special-purpose compute modules on a single
board. While this design is increasingly common, support for it in embedded
software is entirely nonexistent, with inter-microcontroller communication
recreated in an ad hoc fashion for each platform. A need has arisen for
software primitives that support message passing, task migration, and platform
management in a principled manner[cite]. Open-Source Research. Research,
especially publicly funded research, should benefit the public. I am committed
to making my research publicly available. This means not just the papers, but
the hardware and software designs as well. Whenever possible, my first step in
starting a new research project is creating a public Github repo for it.
Platforms research is particularly amenable to this goal, and all hardware and
software designs for both Signpost and Tock are publicly available.

\textbf{\textsf{\large Mentoring Undergraduate Research.}}\\
Working with undergraduate researchers has always been a significant part of my
research goals. My own experiences as an undergraduate researcher taught me
both engineering and communications skills that serve me to this day, and I
want to provide other students with that opportunity. In my mind, mentorship is
part of the goal of academic research. At Michigan, I advised nine
undergraduates and two high school students working on various projects. Five
were included as authors on lab publications (mostly demo and workshop papers,
but also two conference papers) and four have gone on to pursue PhDs in
computer science, including two women. I see my role as a mentor to encourage
these students to explore new domains and build deeper expertise. The most
important thing is for them to have a clear goal for what skills they're hoping
to learn in their research, then we determine how we can fit those goals to
active research projects. Of particular interest to me moving forward is
supporting undergraduate teams in research and engineering efforts.


%%%%%%%%%%%%%%%%%%%%%%%%%%%%%%%%%%%%%%%%%%%%%%%%%%%%%%%%%%%%%%%%%%%%%%%%%%%%%%%%%%


\renewcommand*{\bibfont}{\footnotesize}

% https://tex.stackexchange.com/questions/22645/hiding-the-title-of-the-bibliography
\begingroup
\renewcommand{\section}[2]{}%

% title first commands
\nameusesave
\nameuse{false}

\medskip
\textbf{\textsf{\large References}}

%\bibliographystyle{plain}
%\bibliography{references}
\printbibliography

\nameuserestore

\endgroup

\end{document}

